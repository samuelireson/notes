\chapter{Topological Spaces}
\section{Topologies}
\begin{definition}[Topology]
	For a set $ X $, a \defined{topology on $ X $} is a collection $ \tau $ of subsets of $ X $ which satisfies,
	\begin{enumerate}
		\item $ \varnothing, X \in \tau $.
		\item $ U_{1},...,U_{n}\in \tau \implies U_{1}\cap \cdots \cap U_{n} \in \tau $.
		\item $ ( U_{\alpha} )_{\alpha \in A} \in \tau \implies \bigcup_{\alpha \in A}^{}{U_{\alpha}} \in \tau $.
	\end{enumerate}
	The pair $ ( X, \tau ) $ is called a \defined{topological space}.
\end{definition}

\begin{definition}[Neighbourhood]
	Given a topological space $ X $, a \defined{neighbourhood} of a point $ p \in X $ is any open set $ U_{p} \in \tau $ containing $ p $.

	A \defined{neighbourhood of the subset} $ K \subseteq X $ is an open set containing $ K $.
\end{definition}

\begin{proposition}[Metric topology]
	Let $ ( M, d ) $ be a metric space and let $ \tau $ be the collection of sets which are open in the sense of metric spaces. Then, $ ( M, \tau ) $ is a topological space.
	\begin{proof}
		Proof of the statement amounts to showing that each of the defining properties of a topological space are true in the metric space.

		\begin{enumerate}
			\item $ \varnothing $ is vacuously open in $ M $, and $ M $ is clearly also open. Hence $ \varnothing, M \in \tau $.
			\item Take $ U_{1}, ..., U_{n} \in \tau $, and consider a point $ p \in U_{1} \cap $. Since each of the sets $ U_{i} $ are open, there exist values $ r_{i} $ such that $ B ( p; r_{i} ) \subseteq U_{i} $ for all $ 1 \leq i \leq n $. Taking $ r = \min_{1 \leq i \leq n}\{ r_{i} \} $, we have that $ B ( p; r )\subseteq U_{1}\cap \cdots \cap U_{n} $, and hence this intersection is open.
			\item For every point in the union, there is at least one open set $ U_{\alpha} $ containing the point, and an open ball around the point. Hence the union is open.
		\end{enumerate}
	\end{proof}
\end{proposition}

\begin{notation}
	We have the following standard notations for common sets.
	\begin{itemize}
		\item The open unit $ n $-ball:
		      \begin{align*}
			      \mathbb{B}^{n} = \{ x \in \mathbb{R}^{n}: | x | < 1 \}
		      \end{align*}
		\item The closed unit $ n $-ball:
		      \begin{align*}
			      \overline{\mathbb{B}}^{n} = \{ x \in \mathbb{R}^{n}: | x | \leq 1 \}
		      \end{align*}
		\item The unit $ n $-sphere:
		      \begin{align*}
			      \mathbb{S}^{n} = \{ x \in \mathbb{R}^{n+1}: | x | = 1 \}
		      \end{align*}
	\end{itemize}
\end{notation}

\begin{definition}[Metrizable]
	A topological space $ ( X, \tau ) $ is said to be \defined{metrizable} if its topology is the same as that generated by some metric on $ X $.
\end{definition}

\subsection{Closed subsets}
\begin{definition}[Closed sets]
	A subset $ F $ of a topological space $ X $ is said to be \defined{closed} if its complement $ X \setminus F $ is open.
\end{definition}

For the proceeding four definitions, let $ A $ be any subset of a topological space $ X $.

\begin{definition}[Closure]
	The \defined{closure} of $ A $  in $ X $ is defined to be,
	\begin{align*}
		\overline{A} = \bigcap{\{ B \subseteq X: B \supseteq A \text{ and } B \text{ is closed in } X \}}
	\end{align*}
\end{definition}

\begin{definition}[Interior]
	The \defined{interior} of $ A $ is defined to be,
	\begin{align*}
		\interior{A} = \bigcup{\{ C \subseteq X: C \subseteq A \text{ and } C \text{ is open in } X \}}
	\end{align*}
\end{definition}

\begin{definition}[Exterior]
	The \defined{exterior} of $ A $ is defined to be,
	\begin{align*}
		\exterior{A} = X \setminus \overline{A}
	\end{align*}
\end{definition}

\begin{definition}[Boundary]
	The \defined{boundary} of $ A $ is defined to be,
	\begin{align*}
		\boundary{A} = X \setminus ( \interior{A} \cup \exterior{A} )
	\end{align*}
\end{definition}

\begin{proposition}
	Let $ X $ be a topological space and let $ A \subseteq X $ be any subset.
	\begin{enumerate}
		\item A point is in $ \interior{A} $ if and only if it has a neighbourhood contained in $ A $.
		\item A point is in $ \exterior{A} $ if and only is it has a neighbourhood contained in $ X \setminus A $.
		\item A point is in $ \boundary{A} $ if and only if every neighbourhood of it contains both a point of $ A $ and a point of $ X \setminus A $.
		\item A point is in $ \overline{A} $ if and only if every neighourhood of it contains a point of $ A $.
		\item $ \overline{A} = A \cup \boundary{A} = \interior{A} \cup \boundary{A} $.
		\item $ \interior{A} $ and $ \exterior{A} $ are open in $ X $, while $ \overline{A} $ and $ \boundary{A} $ are closed in $ X $.
		\item The following are equivalent:
		      \begin{itemize}
			      \item $ A $ is open in $ X $.
			      \item $ A = \interior{A} $.
			      \item $ A $ contains none of its boundary points.
			      \item Every point of $ A $ has a neighbourhood contained in $ A $.
		      \end{itemize}
		\item The following are equivalent:
		      \begin{itemize}
			      \item $ A $ is closed in $ X $.
			      \item $ A = \overline{A} $.
			      \item $ A $ contains all of its boundary points.
			      \item Every point of $ X \setminus A $ has a neighbourhood contained in $ X \setminus A $.
		      \end{itemize}
	\end{enumerate}
	\begin{proof}
		We will work through each statement in turn.
		\begin{enumerate}
			\item Consider the following logical equivalence
			      \begin{gather*}
				      x \in \interior{A}                                       \\
				      \iff                                                     \\
				      x \in \bigcup_{}^{}{\{ B \subseteq A: B \in \tau_{X} \}} \\
				      \iff                                                     \\
				      \exists B \subseteq A \text{ s.t. } ( x \in B \land B \in \tau_{X}).
			      \end{gather*}
			      This is exactly the statement that there exists an open neighbourhood of $ x \in \interior{A} $ contained in $ \interior{A} $.
			\item Assume first that $ x \in \exterior{A} $. Then, by definition, we have that $ x \in X \setminus \overline{A} $. Since $ \overline{A} $ is an intersection of closed sets, it is closed, and hence $ X \setminus \overline{A} $ is open. Furthermore, $ A \subseteq \overline{A} $, and equivalently, $ X \setminus \overline{A}\subseteq X \setminus A $. Therefore, $ X \setminus \overline{A} $ is a neighbourhood of $ x $ satisfying the desired property.

			      For the reverse implication, assume that $ x $ has an open neighbourhood $ U \subseteq X \setminus A $. Since $ U $ is open, $ X \setminus U $ is closed, and furthermore, $ X \setminus U \supseteq A $. From the definition of $ \overline{A} $, it must be the case that $ x \not\in \overline{A} $, hence $ x \in \exterior{A} $.

			\item Given the previous two results, and the definition of $ \boundary{A} $, the statement is clear.

			\item Given the second result, and the definition of $ \exterior{A} $, the statement is clear.

			\item We first aim to show that $ A \cup \boundary{A} \subseteq \overline{A} $. Considering each of the LHS components in turn, we will consider neighbourhoods of points. Firstly, for $ x \in A $, any neighbourhood of $ x $ will contain $ x $ itself, and hence $ x \in \overline{A} $. Secondly, for $ x \in \boundary{A} $, we know by the third result that every neighbourhood contains a point of $ A $ (and $ X \setminus A $), therefore $ x \in \overline{A} $ also.

			      By the definition of $ \interior{A} $ we also have that $ \interior{A} \subseteq A $ -- the interior is a union of open sets contained by $ A $. Therefore, so far we have,
			      \begin{align*}
				      \interior{A} \cup \boundary{A} \subseteq A \cup \boundary{A} \subseteq \overline{A}
			      \end{align*}

			      We can also show by elementary set operations that,
			      \begin{align*}
				      \interior{A} \cup \boundary{A} & = \interior{A} \cup ( X \setminus ( \interior{A} \cup \exterior{A} ) ) \\
				                                     & = X \setminus \exterior{A}                                             \\
				                                     & = \overline{A}
			      \end{align*}

			      And hence the chain of inclusions is a chain of equalities, as needed.

			\item Clearly $ \interior{A} $ is open, as a union of open sets. The same follows for the closedness of $ \overline{A} $ as an intersection of closed sets. As $ \exterior{A} $ is the $ X $-complement of $ \overline{A} $, it is open. As $ \boundary{A} $ is the $ X $-complement of a union of open sets, it is closed.

			\item We first assume that $ A $ is open. Then clearly, every point $ x \in A $ has an open neighbourhood contained in $ A $, since $ A \subseteq A $. From this we know that $ \boundary{A} = \varnothing $ from the third result. From this we know that, $ A = \interior{A} $ since,
			      \begin{gather*}
				      (A \cup \boundary{A} = \interior{A} \cup \boundary{A}) \land ( A \cap \boundary{A} = \varnothing )\\
				      \iff \\
				      A = \interior{A}
			      \end{gather*}
			      also relying on the disjointness of $ \interior{A} $ and $ \boundary{A} $.

			      If $ A = \interior{A} $, then $ A $ is open, since $ \interior{A} $ is open. From this we have a full circle of equivalences.

			\item These equivalences are easily validated, in a similar way as to the previous result.
		\end{enumerate}
	\end{proof}
\end{proposition}

\begin{definition}[Limit and isolated points]
	Given a topological space $ X $, and a set $ A \subseteq X $, we say that a point $ p \in X $ is a \defined{limit point} of $ A $ if every neighbourhood of $ p $ contains a point of $ A $ other than $ p $.

	On the other hand, a point $ p \in A $ is called an \defined{isolated point} of $ A $ if there exists a neighbourhood $ U $  of $ p $ such that $ U \cap A = \{ p \} $.
\end{definition}

\begin{exercise}
	\begin{problem}
	Show that a subset is closed if and only if it contains all of its limit points.
	\end{problem}
	\begin{solution}
		We can make a neat argument using the previous proposition. Since every neighbourhood of every limit point $ x $ of $ A $ contains a point of $ A $ other than the point $ x $ itself, the point $ x $ must be contained by the closure $ \overline{A} $. Furthermore since, $ A $ is closed if and only if $ A = \overline{A} $, every limit point is contained by $ A $ if and only if $ A = \overline{A} $ if and only if $ A $ is closed.
	\end{solution}
\end{exercise}

\begin{definition}[Dense]
	A subset $ A $ of a topological space $ X $ is said to be \defined{dense in $ X $} if $ \overline{A} = X $.
\end{definition}

\begin{exercise}
	\begin{problem}
	Show that a subset $ A \subseteq X $ is dense in $ X $ if and only if every nonempty open subset of $ X $ contains a point in $ A $.
	\end{problem}
	\begin{solution}
		$ \overline{A} = X $ if and only if every neighbourhood of every point in $ X $ has nonempty intersection with $ A $. In particular, the statement `every neighbourhood of every point in $ X $' is exhaustive of nonempty open sets in the topology of $ X $, and hence the statement follows.
	\end{solution}
\end{exercise}

\section{Convergence and continuity}
\begin{definition}[Sequence convergence]
	For a topological space $ X $ and a sequence $ ( x_{n} )_{n=1}^{\infty} $ of points in $ X $, we say that the \defined{sequence converges} to $ x $ if for every neighbourhood $ U $ of $ x $, there exists some $ N \in \mathbb{N} $ such that $ x_{n} \in U $ for all $ n \geq N $.
\end{definition}

\begin{exercise}
	\begin{problem}
	Show that in a metric space, the topological definition of convergence is equivalent to the metric space definition.
	\end{problem}
	\begin{solution}
		To solve this exercise, we want to show that a sequence converges in a metric space if and only if the sequence converges in the respective topological space with the induced metric topology. Consider the following equivalences, starting with the definition of convergence in a metric space.
		\begin{gather*}
			\forall \epsilon>0 \ \exists N \in \mathbb{N} \text{ s.t. } d ( x_{n}, x )< \epsilon \ \forall n \geq N\\
			\iff \\
			\forall \epsilon>0 \ \exists N \in \mathbb{N} \text{ s.t. } x_{n} \in B^{(d)}( x;\epsilon ) \ \forall n \geq N\\
			\iff \\
			\text{ for every neighbourhood } U \text{ of } x \ \exists N \in \mathbb{N} \text{ s.t. } x_{n}\in U \ \forall n \geq N
		\end{gather*}
	\end{solution}
\end{exercise}

\begin{exercise}
	\begin{problem}
	For a topological space $ X $, a subset $ A $ and a sequence $ ( x_{i} )\in A $, show that $ x = \lim_{i \to \infty}{x_{i}} \in \overline{A} $.
	\end{problem}
	\begin{solution}
		Suppose for the sake of contradiction that this wasn't the case, and the limit $ x $ of a convergent sequence $ ( x_{i} ) $ was such that $ x \in \exterior{A} $. Then, by a previous result, there exists an open subset $ U \subseteq X \setminus A $ such that $ x \in U $. Since $ x_{i}\in A $ for all $ i $, $ x_{i} \not \in U $ for all $ i $, and hence the sequence cannot be convergent; the desired contradiction.
	\end{solution}
\end{exercise}

\begin{definition}[Continuity]
	If $ X $ and $ Y $ are topological spaces, a map $ f: X \to Y $ is said to be \defined{continuous} if for every open subset $ U \subseteq Y $, its preimage $ f ^{-1}( U ) $ is open in $ X $.
\end{definition}

\begin{proposition}
	A map between topological spaces is continuous if and only if the preimage of every closed subset is closed.
	\begin{proof}
		Consider the following equivalences.
		\begin{gather*}
			f \text{ is continuous }\\
			\iff \\
			f ^{-1}( B ) \text{ is open, for all } B \text{ open }\\
			\iff \\
			f ^{-1}( Y \setminus A ) \text{ is open, for all } A \text{ closed }\\
			\iff \\
			f ^{-1}( Y )\setminus f ^{-1}( A ) \text{ is open, for all } A \text{ closed }\\
			\iff \\
			X \setminus f ^{-1}( A )\text{ is open, for all } A \text{ closed }\\
			\iff \\
			f ^{-1}( A ) \text{ is closed, for all } A \text{ closed}.
		\end{gather*}
	\end{proof}
\end{proposition}

\begin{proposition}
	Let $ X, Y $ and $ Z $ be topological spaces.
	\begin{enumerate}
		\item Every constant map $ f:X \to Y $ is continuous.
		\item The identity map $ \mathrm{Id}_{X}:X \to X $ is continuous.
		\item If $ f:X \to Y $ is continuous, so is the restriction of $ f $ to any open subset of $ X $.
		\item If $ f:X \to Y $ and $ g: Y \to Z $ are continuous, then so is their composition $ g \circ f: X \to Z $.
	\end{enumerate}
	\begin{proof}
		We deal with each of the statements in turn.
		\begin{enumerate}
			\item Suppose that $ f:X \to Y: x \mapsto a \in Y $. Then, an open set $ U \subseteq Y $ is either $ U \supseteq \{ a \} $ or $ U \not \supseteq \{ a \} $. In the first case, $ f ^{-1}( U ) = X $ which is open, and in the second case $ f ^{-1}( U ) = \varnothing $, which is also open.
			\item Let $ U $ be an open set in $ X $. Then $ f ^{-1}( U ) = U $, open by hypothesis.
			\item This is clear.
			\item Consider an open set $ U \subseteq Z $. Then,
			      \begin{align*}
				      ( g \circ f )^{-1}( U ) & = (f ^{-1}\circ g ^{-1})( U ) \\
				                              & = f ^{-1}( g ^{-1}( U ) )
			      \end{align*}
			      where the continuity of each of these functions ensures that the composition is continuous.
		\end{enumerate}
	\end{proof}
\end{proposition}

\begin{proposition}[Local criterion for continuity]
	A map $ f:X \to Y $ is continuous if and only if each point of $ X $ has a neighbourhood on which the restriction of $ f $ is continuous.
	\begin{proof}
		If the function $ f $ is continuous, then we can simply take $ X $ to be the open neighbourhood of every point.

		On the contrary, let $ U \subseteq Y $, where we aim to show that $ f ^{-1}( U ) $ is open in $ X $. Taking some $ x \in f ^{-1}( U ) $, we know that there exists neighbourhood $ V_{x} $ of $ x $ such that $ f |_{V_{x}} $ is continuous. In particular, $ f|_{V_{x}}^{-1}( U ) $ is open. Also,
		\begin{align*}
			f|_{V_{x}}^{-1}( U ) = \{ x \in V_{x}: f ( x ) \in U \} = V_{x} \cap f ^{-1}( U )
		\end{align*}
		and in particular, $ f|_{V_{x}}^{-1}( U )\subseteq f ^{-1}( U ) $, is an open neighbourhood of $ x $. The arbitrary choice of $ x \in f ^{-1}( U ) $ then determines that $ f ^{-1}( U ) $ is open.
	\end{proof}
\end{proposition}

\begin{definition}[Homeomorphism]
	A \defined{homeomorphism} from $ X $ to $ Y $ is a bijective map $ \phi:X \to Y $ such that $ \phi $ and $ \phi ^{-1} $ are continuous.

	If there exists a homeomorphism between $ X $ and $ Y $, we say that $ X $ and $ Y $ are \defined{topologically equivalent}, or \defined{homeomorphic}. We denote this relation by $ X \homeomorphic Y $.
\end{definition}

\begin{exercise}
	\begin{problem}
	Show that homeomorphisms provide an equivalence relation on topological spaces.
	\end{problem}
	\begin{solution}
		We aim to show that homeomorphisms are reflexive, transitive and symmetric.
		\begin{enumerate}
			\item Reflexive: taking $ \identity_{X}:X \to X $, we see that this map is bijective, continuous, and has continuous inverse. Hence $ X \homeomorphic X $.
			\item Transitive: consider the homeomorphisms $ f:X \to Y $ and $ g:Y \to Z $. Then, $ g \circ f:X \to Z $ is continuous, bijective from $ X $ to $ Z $, and has continuous inverse $ f ^{-1}\circ g ^{-1}: Z \to X $. Hence $ X \homeomorphic Z $.
			\item Symmetric: taking $ f ^{-1} $ as the homeomorphism between $ Y $ and $ X $ is sufficient. Hence $ Y \homeomorphic X $.
		\end{enumerate}
	\end{solution}
\end{exercise}

\begin{exercise}
	\begin{problem}
	Let $ ( X_{1}, \tau_{1} ) $ and $ ( X_{2}, \tau_{2} ) $ be topological spaces and let $ f:X_{1}\to X_{2} $ be a bijective map. Show that $ f $ is a homeomorphism if and only if $ f ( \tau_{1} ) = \tau_{2} $ in the sense that $ U \in \tau_{1} $ if and only if $ f ( U ) \in \tau_{2} $.
	\end{problem}
	\begin{solution}
		First, assume that $ f ( \tau_{1} ) = \tau_{2} $, in the sense described. Then $ V \in \tau_{2} $ if and only if there exists some $ U \in \tau_{1} $ such that $ f ( U ) = V $. Since $ f $ is bijective by assumption,
		\begin{align*}
			f ^{-1}( V ) = f ^{-1}( f ( U ) ) = U
		\end{align*}
		is open, and therefore $ f $ is continuous. A similar argument follows for the continuity of $ f ^{-1} $.

		On the contrary, suppose that $ f $ is a homeomorphism. Then, $ f $ and $ f ^{-1} $ are continuous, and,
		\begin{gather*}
			U \in \tau_{1}\\
			\implies \\
			( f ^{-1} )^{-1}( U ) = f ( U ) \in \tau_{2}\\
			\implies \\
			f ^{-1}( f ( U ) ) = U \in \tau_{1}
		\end{gather*}
		which is the statement we wanted.
	\end{solution}
\end{exercise}

\begin{definition}[Finer and coarser]
	Given two topologies $ \tau_{1}, \tau_{2} $ on $ X $, we say that $ \tau_{1} $ is \defined{finer} than $ \tau_{2} $ if $ \tau_{1} \supseteq \tau_{2} $ and \defined{coarser} than $ \tau_{2} $ if $ \tau_{1} \subseteq \tau_{2} $.
\end{definition}

\begin{exercise}
	\begin{problem}
	Show that the identity map of $ X $ is continuous as a map from $ ( X, \tau_{1} ) $ to $ ( X, \tau_{2} ) $ if and only if $ \tau_{1} $ is finer than $ \tau_{2} $, and is a homeomorphism if and only if $ \tau_{1} = \tau_{2} $.
	\end{problem}
	\begin{solution}
		Considering the map $ \identity_{X}:( X, \tau_{1} )\to ( X, \tau_{2} ) $, we have the following equivalences,
		\begin{gather*}
			\tau_{2} \subseteq \tau_{1}                                  \\
			\iff                                                         \\
			U \in \tau_{2} \implies U \in \tau_{1}                       \\
			\iff                                                         \\
			U \in \tau_{2} \implies \identity_{X}^{-1}( U ) \in \tau_{1} \\
			\iff                                                         \\
			\identity_{X} \text{ is continuous }.
		\end{gather*}

		The identity is bijective, so is a homeomorphism if and only if it is continuous with continuous inverse. This is the case if and only if $ \tau_{1} \subseteq \tau_{2} $ and $ \tau_{2} \subseteq \tau_{1} $, that is if and only if $ \tau_{1} = \tau_{2} $.
	\end{solution}
\end{exercise}

\begin{definition}[Open and closed maps]
	A function $ f:X \to Y $ is said to be an \defined{open map} if it takes open subsets of $ X $ to open subsets of $ Y $.

	A function $ f:X \to Y $ is said to be a \defined{closed map} if it takes closed subsets of $ X $ to closed subsets of $ Y $.
\end{definition}

\begin{exercise}
	\begin{problem}
	Suppose that $ f:X \to Y $ is a bijective continuous map. Show that the following are equivalent.
	\begin{enumerate}
		\item $ f $ is a homeomorphism.
		\item $ f $ is an open map.
		\item $ f $ is a closed map.
	\end{enumerate}
	\end{problem}
	\begin{solution}
		Given the assumptions that $ f $ is continuous and bijective,
		\begin{gather*}
			f \text{ is a homeomorphism}                 \\
			\iff                                         \\
			f ^{-1} \text{ is continuous}                \\
			\iff                                         \\
			U \in \tau_{X} \implies f ( U ) \in \tau_{Y} \\
			\iff                                         \\
			f \text{ is open}.
		\end{gather*}
	\end{solution}
\end{exercise}

\begin{proposition}
	Let $ f:X \to Y $ be a map of topological spaces.
	\begin{enumerate}
		\item $ f $ is continuous if and only if $ f ( \overline{A} ) \subseteq \overline{f ( A )} $ for all $ A \subseteq X $.
		\item $ f $ is closed if and only if $ f ( \overline{A} )\supseteq \overline{f ( A )} $ for all $ A \subseteq X $.
		\item $ f $ is continuous if and only if $ f ^{-1}( \interior{B} )\subseteq \interior{f ^{-1}( B )} $ for all $ B \subseteq Y $.
		\item $ f $ is open if and only if $ f ^{-1}( \interior{B} )\supseteq \interior{f ^{-1}( B )} $ for all $ B \subseteq Y $.
	\end{enumerate}
	\begin{proof}
		\begin{enumerate}
			\item If $ f $ is closed, then $ f ( \overline{A} ) $ is closed. Since $ A \subseteq \overline{A} $, $ f ( A )\subseteq f ( \overline{A} ) $. The closure of $ f ( A ) $ is defined as,
			      \begin{align*}
				      \overline{f ( A )} = \bigcap_{B \supseteq f ( A )}{\{ B: X \setminus B \in \tau_{X} \}}.
			      \end{align*}
			      We know that $ f ( \overline{A} ) $ is a closed set containing $ f ( A ) $, and the result follows.
		\end{enumerate}
	\end{proof}
\end{proposition}

\begin{definition}[Local homeomorphism]
	A map $ f:X \to Y $ is called a \defined{local homeomorphism} if every point $ x \in X $ has a neighbourhood $ U \subseteq X $ such that $ f ( U ) $ is an open subset of $ Y $ and $ f|_{U}:U \to f ( U ) $ is a homeomorphism.
\end{definition}

\begin{proposition}[Properties of local homeomorphisms]
	We have the following properties,
	\begin{enumerate}
		\item Every homeomorphism is a local homeomorphism.
		\item Every local homeomorphism is continuous and open.
		\item Every bijective local homeomorphism is a homeomorphism.
	\end{enumerate}
	\begin{proof}
		We work through the statements in turn.
		\begin{enumerate}
			\item This is clear. Since $ X $ is open, and $ f ( X ) = Y $, $ f $ is a valid local homeomorphism for all points $ x \in X $.
			\item We know from the local criterion for continuity that a function $ f $ is continuous if and only if each point $ x \in X $ has a neighbourhood $ U $ such that $ f|_{U} $ is continuous. If we consider a local homeomorphism $ f $, we know that this function is a homeomorphism on open neighbourhoods of every points $ x \in X $, and therefore is continuous. We also know that every restriction of $ f $ is open, and therefore $ f $ is open also.
			\item Using a previous result, we know that a continuous, open bijection is a homeomorphism.
		\end{enumerate}
	\end{proof}
\end{proposition}

\section{Hausdorff spaces}
\begin{definition}[Hausdorff]
	A topological space $ X $ is said to be \defined{Hausdorff} if given any two points $ p \neq q \in X $, there exist neighbourhoods $ U, V $ of $ p $ and $ q $ respectively such that $ U \cap V = \varnothing $.
\end{definition}

\begin{exercise}
	\begin{problem}
	Suppose that for every $ p \in X $ there exists a continuous function $ f:X \to \mathbb{R} $ such that $ f ^{-1}( 0 ) = \{ p \} $. Show that $ X $ is Hausdorff.
	\end{problem}
	\begin{solution}
		TODO
	\end{solution}
\end{exercise}

\begin{proposition}
	Let $ X $ be a Hausdorff space.
	\begin{enumerate}
		\item Every finite subset of $ X $ is closed.
		\item If a sequence $ ( p_{i} )\in X $ converges to a limit $ p \in X $, the limit is unique.
	\end{enumerate}
	\begin{proof}
		TODO
	\end{proof}
\end{proposition}

\begin{exercise}
	\begin{problem}
	Show that the only Hausdorff topology on a finite set is the discrete topology.
	\end{problem}
	\begin{solution}
		TODO
	\end{solution}
\end{exercise}

\begin{proposition}
	Suppose $ X $ is a Hausdorff space and $ A \subseteq X $. If $ p \in X $ is a limit point of $ A $, then every neighbourhood of $ p $ contains infinitely many points of $ A $.
	\begin{proof}
		TODO
	\end{proof}
\end{proposition}

\section{Bases and countability}
\begin{definition}[Basis]
	A collection $ \mathcal{B} $ of subsets of $ X $ is a \defined{basis for the topology} of $ X $ if,
	\begin{enumerate}
		\item Every element of $ \mathcal{B} $ is an open set of $ X $. That is $ \mathcal{B} \subseteq \tau $.
		\item Every open subset of $ X $ is the union of some collection of the elements of $ \mathcal{B} $.
	\end{enumerate}
\end{definition}

\begin{exercise}
	\begin{problem}
	Suppose that $ \mathcal{B} $ is a basis for $ X $. Show that a subset $ U \subseteq X $ is open if and only if for each $ p \in U $, there exists $ B \in \mathcal{B} $ such that $ p \in B \subseteq U $.
	\end{problem}
	\begin{solution}
		TODO
	\end{solution}
\end{exercise}

\begin{proposition}
	Let $ X $ and $ Y $ be topological spaces and $ \mathcal{B} $ a basis for $ Y $. A map $ f:X \to Y $ is continuous if and only if for every basis subset $ B \in \mathcal{B} $, the subset $ f ^{-1}( B ) $ is open in $ X $.
	\begin{proof}
		TODO
	\end{proof}
\end{proposition}

\subsection{Defining a topology from a basis}
\begin{proposition}
	A collection $ \mathcal{B} $ of open subsets of $ X $ is a basis if and only if the following two conditions hold,
	\begin{enumerate}
		\item $ \bigcup_{B \in \mathcal{B}}^{}{B} = X $.
		\item If $ B_{1}, B_{2} \in \mathcal{B} $ and $ x \in B_{1}\cap B_{2} $, there exists an element $ B_{3}\in \mathcal{B} $ such that $ x \in B_{3} \subseteq B_{1}\cap B_{2} $.
	\end{enumerate}
	If this is the case, there is a unique topology on $ X $ for which $ \mathcal{B} $ is a basis, called the \defined{generated topology} with respect to $ \mathcal{B} $.
	\begin{proof}
		TODO
	\end{proof}
\end{proposition}

\subsection{Countability properties}
\begin{definition}[Neighbourhood basis]
	If $ X $ is a topological space and $ p \in X $, a collection $ \mathcal{B}_{p} $ of neighbourhoods of $ p $ is called a \defined{neighbourhood basis} for $ X $ at $ p $ if every neighbourhood of $ p $ contains some $ B \in \mathcal{B}_{p} $.
\end{definition}

\begin{definition}[First countability]
	We say that a topological space $ X $ is \defined{first countable} if there exists a countable neighbourhood basis at all points $ x \in X $.
\end{definition}

\begin{definition}[Nested neighbourhood basis]
	If $ X $ is a topological space and $ p \in X $, a sequence $ ( U_{n} )_{n \in \mathbb{N}} $ of neighbourhoods of $ p $ is called a \defined{nested neighbourhood basis} if $ U_{n+1}\subseteq U_{n} $ for each $ n $, and the sequence, when viewed as a collection if a neighbourhood basis of $ X $ at $ p $.
\end{definition}

\begin{lemma}[Nested neighbourhood basis lemma]
	Let $ X $ be a first countable space. Then there exists a nested neighbourhood basis at $ p $, for every $ p \in X $.
	\begin{proof}
		TODO
	\end{proof}
\end{lemma}

\begin{definition}[Eventually in]
	If $ ( x_{i} )_{i=1}^{\infty} $ is a sequence of points in the topological space $ X $ and $ A \subseteq X $, we say that the sequence is \defined{eventually in} $ A $ if there exists some $ n \in \mathbb{N} $ such that $ x_{i} \in A $ for all $ i \geq n $.
\end{definition}

\begin{lemma}
	Let $ X $ be first countable, $ A $ be any subset of $ X $, and $ x \in X $.
	\begin{enumerate}
		\item $ x \in \overline{A} $ if and only if $ x $ is a limit of a sequence of points in $ A $.
		\item $ x \in \interior{A} $ if and only if every sequence in $ X $ converging to $ x $ is eventually in $ A $.
		\item $ A $ is closed in $ X $ if and only if $ A $ contains every limit of every convergent sequence of points in $ A $.
		\item $ A $ is open in $ X $ if and only if every sequence in $ X $ converging to a point of $ A $ is eventually in $ A $.
	\end{enumerate}
	\begin{proof}
		TODO
	\end{proof}
\end{lemma}

\begin{example}[A non-first countable space]
	TODO
\end{example}

\begin{definition}[Second countability]
	A topological space is said to be \defined{second countable} if it admits a countable basis for its topology.
\end{definition}

\begin{definition}[Covers]
	A collection $ \mathcal{U} $ of subsets of $ X $ is called a \defined{cover} of $ X $ if every points $ x \in X $ is contained by at least one $ U \in \mathcal{U} $. The cover is called an \defined{open cover} if every $ U \in \mathcal{U} $ is open, and a \defined{closed cover} if every $ U \in \mathcal{U} $ is closed.

	Given a cover $ \mathcal{U} $, a \defined{subcover} of $ \mathcal{U} $ is a subcollection $ \mathcal{U}' \subseteq \mathcal{U} $ which covers $ X $.
\end{definition}

\begin{definition}[Separable]
	A topological space is called \defined{separable} if it contains a countable dense subset.
\end{definition}

\begin{definition}[Lindel\"of]
	A topological space, $ X $ is said to be a \defined{Lindel\"of space} if every open cover of $ X $ has a countable subcover.
\end{definition}

\begin{theorem}[Properties of second countable spaces]
	Let $ X $ be a second countable space.
	\begin{enumerate}
		\item $ X $ is first countable.
		\item $ X $ is separable.
		\item $ X $ is Lindel\"of.
	\end{enumerate}
	\begin{proof}
		TODO
	\end{proof}
\end{theorem}

\section{Manifolds}
\begin{definition}[Locally Euclidean]
	A topological space $ M $ is called \defined{locally Euclidean} of dimension $ n $ if every point of $ M $ has a neighbourhood in $ M $ that is homeomorphic to an open subset of $ \mathbb{R}^{n} $.
\end{definition}

\begin{lemma}
	A topological space $ M $ is locally Euclidean of dimension $ n $ if and only if either of the following properties hold:
	\begin{enumerate}
		\item Every points of $ M $ has a neighbourhood homeomorphic to an open ball in $ \mathbb{R}^{n} $.
		\item Every points of $ M $ has a neighbourhood homeomorphic to $ \mathbb{R}^{n} $.
	\end{enumerate}
	\begin{proof}
		TODO
	\end{proof}
\end{lemma}

\begin{definition}[Topological manifold]
	An $ n $-dimensional \defined{topological manifold} is a secound countable Hausdorff space which is locally Euclidean of dimension $ n $.
\end{definition}

\begin{proposition}
	Every open subset of an $ n $-manifold is an $ n $-manifold.
	\begin{proof}
		TODO
	\end{proof}
\end{proposition}

\begin{exercise}
	\begin{problem}
	Show that a topological space is a $ 0 $-manifold if and only if it is a countable discrete space.
	\end{problem}
	\begin{solution}
		TODO
	\end{solution}
\end{exercise}

\begin{proposition}
	A separable metric space that is locally Euclidean of dimension $ n $ is an $ n $-manifold.
\end{proposition}

\subsection{Manifolds with boundary}
This content is basically the same, just slightly uglier. I will skip the sections on manifolds with boundary as they come up.

\section{Problems}


